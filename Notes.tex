\documentclass[12pt]{article}

\usepackage[table]{xcolor}
\usepackage{amsmath, amsthm}
\usepackage{amsfonts,amssymb}
\usepackage{hyperref}
\usepackage{authblk}
\usepackage[all]{xy}
\usepackage{enumerate}
\usepackage{array}
\usepackage{makecell}
\usepackage{lscape}
\usepackage{dsfont}

\newcommand{\N}{\mathbb{N}}
\newcommand{\Z}{\mathbb{Z}}
\newcommand{\Q}{\mathbb{Q}}
\newcommand{\F}{\mathbb{F}}
\renewcommand{\P}{\mathbb{P}}
\newcommand{\R}{\mathbb{R}}
\newcommand{\C}{\mathbb{C}}
\newcommand{\G}{\mathbb{G}}

\newcommand{\M}{\mathcal{M}}
\renewcommand{\S}{\mathcal{S}}
\newcommand{\E}{\mathcal{E}}

\renewcommand{\l}{\mathfrak{l}}

\newcommand{\Gal}{\operatorname{Gal}}
\newcommand{\GQ}{\Gal(\overline \Q / \Q)}
\newcommand{\GL}{\operatorname{GL}}
\newcommand{\SL}{\operatorname{SL}}
\newcommand{\PGL}{\operatorname{PGL}}
\newcommand{\PSL}{\operatorname{PSL}}
\newcommand{\GLFl}{\GL_2(\F_\ell)}
\newcommand{\PGLFl}{\PGL_2(\F_\ell)}
\newcommand{\PSLFl}{\PSL_2(\F_\ell)}
\newcommand{\Ker}{\operatorname{Ker}}
\renewcommand{\Im}{\operatorname{Im}}
\newcommand{\rev}{\operatorname{rev}}
\newcommand{\disc}{\operatorname{disc}}
\newcommand{\Res}{\operatorname{Res}}
\newcommand{\Aut}{\operatorname{Aut}}
\newcommand{\Hom}{\operatorname{Hom}}
\newcommand{\End}{\operatorname{End}}
\newcommand{\Ext}{\operatorname{Ext}}
\newcommand{\ab}{\operatorname{ab}}
\newcommand{\Tra}{\operatorname{Tra}}
\newcommand{\Inf}{\operatorname{Inf}}
\newcommand{\ord}{\operatorname{ord}}
\newcommand{\Frob}{\operatorname{Frob}}
\newcommand{\sgn}{\operatorname{sgn}}
\newcommand{\matabcd}{\left( \begin{smallmatrix} a & b \\ c & d \end{smallmatrix} \right)}

\newcommand{\TableMinPolyVar}{x}


\newcommand{\smat}[4]{\left[ \begin{smallmatrix} #1 & #2 \\ #3 & #4 \end{smallmatrix} \right]}
\newcommand{\mat}[4]{\left[ \begin{matrix} #1 & #2 \\ #3 & #4 \end{matrix} \right]}

\newtheorem{thm}{Theorem}
\newtheorem{lem}[thm]{Lemma}
\theoremstyle{definition}
\newtheorem{rk}[thm]{Remark}
%\newtheorem{ex}{Example}[thm] % TODO remove ?

\topmargin -2.5cm
\textheight 25cm
\textwidth 15cm

\title{Notes: $p$-adic modular Jacobians via moduli-friendly Eisenstein series}
\author{Nicolas Mascot}

\begin{document}

Fix $N \in \N$. Want to use Kamal's algos in rep B0.


Main idea: fix $E/\Q$, and a naive $\Gamma(N)$-struct (in the sens of Katz76) $\alpha: (\Z/N\Z)^2 \simeq E[N]$. Let $\zeta_N = e_N(\alpha(1,0),\alpha(0,1))$, a prim root. The other structs on $E$ are the $\alpha \circ \gamma$, $\gamma \in \Gamma(N)$ (points $P_\gamma$ in fibre of $X(N) \longrightarrow X(1) \ni [E]$ acting on the \emph{right} on row-vectors $v \in (\Z/N\Z)^2$.

By Kamal, the $f_v : (E,\beta) \mapsto \lambda_{\beta(v)}$ span $\E_1(\Gamma(N))$. If for $P,Q \in E[N]$ distinct and nonzero, we let $\lambda_{P,Q} = $ slope of line joining $P$ and $Q$, then we have $\lambda_{P,Q} = \lambda_P + \lambda_{Q} + \lambda_{R}$ where $P+Q+R=0$. We can thus recover $\lambda_P = \frac1N \sum_{x \bmod N} \lambda_{P,Q+xP}$, so we can evaluate the $f_v$ at the $P_\gamma$. TODO since $\lambda_{P,Q} = -Z_E(P)-Z_E(Q)-Z_E(-P-Q)$, this implies that $\lambda_P$ is proportional to $Z_E(P)$, which in turns implies by DS that up to a mult constant,
\[ f_v = g_1^v = \left. \begin{array}{cc} \frac12 \frac{1+\zeta^d}{1-\zeta^d}, & c =0 \\ \frac12 - \frac{c}N, & c \neq 0 \end{array} \right\} + \sum_{n \geqslant 1} \left( \sum_{\substack{n=ab \\ a,b \in \Z \\ a \equiv c \bmod N}} \sgn(b) \zeta^{bd} \right) q_N^n \]
where $v=(c,d)$. 

Up to a mult constant, the value of $f_v$ at the cusp $(c,d)$ of $X(N)$ is $s(c,d) = 1/2 - \{c/N\}$ if $c \not \equiv 0$, and $\frac12 \frac{\zeta_N^d+1}{\zeta_N^d-1}$ if $c \equiv 0$.

The $f_{v} f_{v'}$ span $\M_2(\Gamma(N))$, so for any $\Gamma \supset \Gamma(N)$, the $\sum_{t \in \Gamma / \Gamma(N)} f_{v t } f_{v' t}$ span $\M_2(\Gamma)$, provided that $p \nmid \# \Gamma/\Gamma(N)$ (this will always be the case if $p \nmid \#\SL_2(\Z/N\Z) = \prod_{l^e \Vert N} l^{4e-3} (l-1)(l+1)$).

Note: This is very general and allows one to work with \emph{any} congruence subgroup.

Special case $\Gamma = \Gamma_H(N)$: Can take
\[ t = \smat{1}{x}{0}{1} \smat{h^{-1}}{*}{N*}{h} \quad (x \bmod N, h \in H) \]
and it is safe if $p \mid N \#H$, so in particular if $p \nmid N \phi(N)$.

\section{Galois action}

\subsection{Cusps}

Let $C_n = (\P^1 \times \Z/n\Z) / (\infty,i) \sim (0, i+1)$ be Neron $n$-gon. We have $C_n \supset C_n^{\text{reg}} = \G_m \times \Z/n\Z$, which has
\[ \End = \left( \begin{matrix} \End \G_m & \Hom(\Z/n\Z,\G_m) \\ \Hom(\G_m, \Z/n\Z) & \End(\Z/n\Z) \end{matrix} \right) =  \left( \begin{matrix} \Z & \mu_n \\ 0 & \Z/n\Z \end{matrix} \right) \]
acting by 
\[ \left( \begin{matrix} m & \zeta \\ 0 & j \end{matrix} \right) \left( \begin{matrix} x \\ i \end{matrix} \right)= \left( \begin{matrix} \zeta^i y^m \\ ji \end{matrix} \right),  \]
whence
\[ \Aut =  \left( \begin{matrix} \pm1 & \mu_n \\ 0 &( \Z/n\Z )^\times \end{matrix} \right). \]
The condition $(\infty,i) \sim (0, i+1)$ forces $j=m$, so finally
\[ \Aut C_n = \pm \left( \begin{matrix} 1 & \mu_n \\ 0 &1 \end{matrix} \right). \]

We can thus define bijection
\[ \text{Cusps}(\Gamma_1(N)) \longleftrightarrow \{ (c,d) \in \Z/N\Z \times (\Z/(c,N)\Z)^\times \} / \pm 1 \]
by taking $a/c$ to $\matabcd \in \SL_2(\Z)$, since the others are the $\pm \left( \begin{smallmatrix} a & b-td \\ c & d+tc \end{smallmatrix} \right)$; then $(c,d)$ corresponds to
\[ \big( C_{N/(c,N)}, (\zeta_N^d, c/(c,N) ) \big). \]

We thus get
\[ \Frob_p (c,d) = \Frob_p \big( C_{N/(c,N)}, (\zeta_N^d, c/(c,N) ) \big) = \big( C_{N/(c,N)}, (\zeta_N^{pd}, c/(c,N) ) \big) = (c,pd) \]
so two cusps are in the same Galois orbit iff. they have the same $c$.

Go down to $\Gamma_H$: let $M_s$ be such that $M_S \cdot \infty = s$. Then for $\gamma \equiv \smat{h^{-1}}{x}{}{h}$,
\[ a/c \sim \gamma \cdot a/c = \gamma M_{a/c} \cdot \infty \]
so
\[ \smat{*}{*}{c}{d} \sim \gamma \smat{*}{*}{c}{d} = \smat{*}{*}{hc}{hd} \]
whence
\[ \text{Cusps}(\Gamma_H(N)) \longleftrightarrow \{ (c,d) \in \Z/N\Z \times (\Z/(c,N)\Z)^\times \} / H. \]

TODO width

\subsection{Forms}

On  q-exps:
\[ \Frob_p g_1^{(c,d)}(q) = g_1^{(c,pd)}(q) \]
so
\[ \sigma_x (g_k^v (q) )= g_k^{v B_x} (x) \]
where $B_x = \smat{1}{0}{0}{x}$.
 
Besides $s=(c=1,d=0)$ is def on $\Q$ so $f$ def on $\Q$ iff. $f \vert M (q) \in \Q[[q]]$.

So given $v_1,\cdots,v_k$, we define
\[ F = \sum_{\substack{x \in (\Z/N\Z)^\times \\ t \in T}} \zeta^x \prod_{j=1}^k g_1^{v_j t \smat{x}{0}{0}{1} } \]
\[ = \sum_{\substack{x \in (\Z/N\Z)^\times \\ t \in T}} \zeta^x \prod_{j=1}^k g_1^{v_j  \smat{x}{0}{0}{1} t} \]
where the equality follows from the fact that  $\smat{x}{0}{0}{1}$ normalises $T$, and we have
\[ F \in M_k(\Gamma;\Q). \]

\section{L factor}

\[ L_p\big(X_H(N)\big) = \hspace{-1cm} \prod_{\substack{\chi \bmod G_\Q \\ \chi : (\Z/N\Z)^\times \rightarrow \Q[t]/\Phi_{\ord \chi}(t) \\ \Ker \chi \geqslant H}} \hspace{-1cm} \Res_t\big( \Phi_{\ord \chi}(t),\Res_y(x^2-yx+p\chi(p), \det(y1-T_p \vert_{S_2(N,\chi)}) \big) \]

\section{Torsion basis}

Let $D_1, D_2 \in J[N]$. Then $e_N(D_1,D_2) = f_1(D_2)/f_2(D_1)$, where $(f_i) = N D_i$. As a result, $e_{ab} = e_a^b$ when this makes sense.

Factor $N = \prod_k l_k^{v_k}$, and for each $k$, let $M_k = N/l_k^{v_k}$, $i_k \in \Z/N\Z$ such that $i_k \bmod l_j^{v_j} = \mathds{1}_{j=k}$, $P_k, Q_k \in E[l_k^{v_k}]$, and $z_k = e_{l_k^{v_k}}(P_k,Q_k)$. Set $P = \sum_k P_k$, $Q = \sum_k Q_k$, which are thus $N$-torsion, and $z=e_N(P,Q)$. Then the $i_k$ are idempotent, $\sum_k i_k = 1$, and
\[ z_k^{M_k} = e_{l_k^{v_k}}(P_k,Q_k)^{M_k} = e_N(P_k,Q_k) = e_N(i_k P, i_k Q) = z ^{i_k^2} = z^{i_k} \]
so that
\[ z = \prod_k z_k^{M_k}. \]
In particular, if $z_k$ is a primitive $l_k^{v_k}$-th root of 1 for each $k$, then $z$ is a primitive $N$-th root of 1, and $(P,Q)$ is a basis of $E[N]$.

We also need the matrix of $\Frob_p$ on $E[N]$. We get it by CRT. Idea: if $R = x P_k + y Q_k \in E[l_k^{v_k}]$, then $e_{l_k^{v_k}}(P_k,R) = z_k^y$ and $e_{l_k^{v_k}}(Q_k,R) = z_k^{-x}$, so the matrix of $\Frob_p$ on $E[l_k^{v_k}]$ is
\[ \left[ \begin{matrix} \log_{z_k} e_{l_k^{v_k}}(Q_k,P_k^{\Frob_p}) & \log_{z_k} e_{l_k^{v_k}}(Q_k,Q_k^{\Frob_p}) \\ -\log_{z_k} e_{l_k^{v_k}}(P_k,P_k^{\Frob_p}) & -\log_{z_k} e_{l_k^{v_k}}(P_k,Q_k^{\Frob_p}) \end{matrix} \right] \]

\section{Finding a suitable curve}

Let $E/\Q$, $q=p^a$, $N$, $a_p=a_p(E)$. Want $\Frob_q = \Frob_p^a$ trivial on $E[N]$. We can suppose $q \equiv 1 \bmod N$.

Factor $N = \prod l_i^{v_i}$, let $\Delta = \disc(x^2-a_p x+p)=(x-\alpha)(x-\beta)$, $M = \prod_{l_i \nmid \Delta} l_i^{v_i}$. Then $\Frob_p$ semisimple on $E[M]$. Let $M_1 = \prod_{l_i \nmid \Delta} l_i$ the radical of $M$, then the charpoly of $\Frob_q$ is $(x-\alpha^q)(x-\beta^q) \equiv x^2-\nu_d x + 1 \bmod M_1$, so $\Frob_q$ trivial on $E[M_1]$ iff. $\nu_d \equiv 2 \bmod M_1$.

If now $l_i \mid \Delta$, then $\Frob_p$ not semisimple on $E[l_i^{v_i}]$, unique eigenvalue $c = a_p/2 \bmod l_i^{v_i}$, so unipotent iff. $c^d \equiv 1 \bmod l_i^{v_i}$.

If furthermore $l_i^{v_i} \mid d$, then this implies $\Frob_p$ trivial on $E[l_i]^{v_i}$.

Let now $l_i^{v_i}$ such that $l_i \mid \Delta$ or $v_i$; we know $\Frob_p$ unipotent on $E[l_i^{v_i}]$, $D=\psi_{l_i^{v_i}}$. If $D$ not tot split over $\F_q$, then $\Frob_q$ nontrivial on $E[l_i^{v_i}]/\pm1$. Else, if $l_i > 2$, check that $x^3+ax+b$ square in $\F_q$ for each root $x$ of $D$ (for $l_i=2$, this is automatic). 



\end{document}
