\documentclass[12pt]{beamer}
\usepackage{etex}
\usepackage{amsmath, amsthm}
\usepackage{amsfonts,amssymb}
\usepackage[utf8]{inputenc}
%\usepackage[english]{babel}
\usepackage[all]{xy}
%\usepackage{xcolor}
\usepackage{graphicx}
\usepackage[usenames,dvipsnames]{pstricks}
\usepackage{bbm}
\usepackage{ skull }

\usepackage{beamerthemeWarsaw}
\usepackage{color}
\setbeamertemplate{itemize item}[ball]
\setbeamertemplate {navigation symbols}{} 
%\definecolor{mauve}{RGB}{100,0,100}
%\setbeamercolor{block title example}{fg=white,bg=mauve}
%\setbeamercolor{block body example}{fg=black,bg=mauve!30}
%\setbeamercolor{itemize item}{fg=mauve}


\newcommand{\A}{\mathbb{A}}
\newcommand{\C}{\mathbb{C}}
\newcommand{\E}{\mathbb{E}}
\newcommand{\F}{\mathbb{F}}
\newcommand{\G}{\mathbb{G}}
\renewcommand{\H}{\mathcal{H}}
\newcommand{\Hdot}{\H^\bullet}
\renewcommand{\L}{\mathcal{L}}
\newcommand{\M}{\mathcal{M}}
\newcommand{\MM}{\mathbb{M}}
\newcommand{\N}{\mathbb{N}}
\newcommand{\p}{\mathfrak{p}}
\renewcommand{\P}{\mathbb{P}}
\newcommand{\Q}{\mathbb{Q}}
\newcommand{\R}{\mathbb{R}}
\renewcommand{\S}{\mathcal{S}}
\renewcommand{\SS}{\mathbb{S}}
\newcommand{\T}{\mathbb{T}}
\newcommand{\Z}{\mathbb{Z}}

\newcommand{\Pic}{\operatorname{Pic}}
\newcommand{\ord}{\operatorname{ord}}
\newcommand{\Div}{\operatorname{Div}}
\newcommand{\Ppal}{\operatorname{Ppal}}
\newcommand{\Jac}{\operatorname{Jac}}
\renewcommand{\Im}{\operatorname{Im}}
\newcommand{\res}{\operatorname{res}}
\newcommand{\charf}{\mathbbm{1}}
\renewcommand{\l}{\mathfrak{l}}
\newcommand{\Ker}{\operatorname{Ker}}
\newcommand{\Gal}{\operatorname{Gal}}
\newcommand{\Tr}{\operatorname{Tr}}
\newcommand{\GL}{\operatorname{GL}}
\newcommand{\PGL}{\operatorname{PGL}}
\newcommand{\PSL}{\operatorname{PSL}}
\newcommand{\SL}{\operatorname{SL}}
\newcommand{\SLZ}{\SL_2(\Z)}
\newcommand{\Fl}{{\F_\ell}}
\newcommand{\Flx}{{\F_\ell^*}}
\newcommand{\GLFl}{\GL_2(\Fl)}
\newcommand{\PGLFl}{\PGL_2(\Fl)}
\newcommand{\PSLFl}{\PSL_2(\Fl)}
\newcommand{\Aut}{\operatorname{Aut}}
\newcommand{\Frob}{\operatorname{Frob}}
\newcommand{\proj}{{\text{proj}}}
\newcommand{\newf}{\mathcal{N}}
\newcommand{\eps}{\varepsilon}
\newcommand{\Ta}{\operatorname{Ta}}
\newcommand{\End}{\operatorname{End}}
\newcommand{\Res}{\operatorname{Res}}
\newcommand{\cond}{\operatorname{cond}}
\newcommand{\ab}{\text{ab}}

%\newtheorem{thm}{Théorème}[subsection]
%\newtheorem{pro}[thm]{Proposition}
%\newtheorem{cor}[thm]{Corollaire}
%\newtheorem{lem}[thm]{Lemme}
%\newtheorem{fait}[thm]{Fait}
%\theoremstyle{definition}
%\newtheorem{de}[thm]{Définition}
%\theoremstyle{definition}
%\newtheorem{ex}[thm]{Exemple}
%\newtheorem{conj}[thm]{Conjecture}

\title[Hensel-lifting torsion points]{Hensel-lifting torsion points \\ and Galois representations}
\author[Nicolas Mascot]{Nicolas Mascot}
\institute{American University of Beirut}
\date{Arithmetic geometry, number theory, and computation \\ M.I.T. \\ August $21^\text{st}$ 2018}
%\titlegraphic{\includegraphics[scale=0.2]{Logo2.jpg}}

\begin{document}

\begin{frame}

\titlepage

\end{frame}

\begin{frame}
\frametitle{Goal}

Let $\displaystyle\rho : \Gal(\overline \Q/\Q) \longrightarrow \GL_d(\Fl)$ be a Galois representation.

\vspace{1cm}

\uncover<2->{Suppose we know a curve $C/ \Q$ such that $\rho$ is afforded by $T \subset J[\ell]$, where $J = \Jac(C)$.}

\vspace{1cm}

\uncover<3->{To isolate $T \subset J[\ell]$, we assume that for one prime $p \in \N$, we know
\[ \chi_\rho(x) = \det\big( t - \rho(\Frob_p) \big) \in \Fl[x] \]
and
\[ L(x) = \det\big( t -\Frob_p \vert_J \big) \in \Z[x], \]
and that
\[ \gcd( \chi_\rho, L/\chi_\rho) = 1 \in \Fl[x]. \]}
\end{frame}

\begin{frame}
\frametitle{Strategy}

\begin{enumerate}
\item Find $q=p^a$ such that $T \subset J(\F_q)[\ell]$,

 \vspace{5mm}

 \item Generate random $\F_q$-points of $T$ until we get a basis,
 
 \vspace{5mm}
 
 \item Lift these points from $J(\F_q)$ to $J(\Q_q)$,
 
  \vspace{5mm}
  
 \item Form all linear combinations of these points,
 
  \vspace{5mm}
  
 \item $F(x) = \prod_{t \in T} \big(x - \alpha(t)\big)$, where $\alpha : J \dashrightarrow \A^1$,
 
  \vspace{5mm}
  
 \item Identify $F(x) \in \Q[x]$.

\end{enumerate}

\end{frame}

\begin{frame}
\frametitle{Getting a basis of T}

\begin{itemize}

\item $\#J(\F_q) = \Res\big(L(x),x^a-1\big) = \ell^b M$.

\[ \leadsto t \in J(\F_q) \Longrightarrow [M]t \in J[\ell^\infty]. \]

\vspace{1cm}

\uncover<2->{\item $L(x) = \chi_\rho(x) \psi(x) \in \Fl[x]$

\[ \leadsto t \in J[\ell] \Longrightarrow \psi (\Frob_p) \cdot t \in T. \]}

\vspace{1cm}

\uncover<3->{\item Linear dependence in $J[\ell]$ can be detected by parings.}

\end{itemize}

\end{frame}

\begin{frame}
\frametitle{Hensel-lifting in $J$}

\begin{block}{Makdisi's algorithms}

\vspace{2mm}

\begin{itemize}

\item Fix $P_1, \cdots, P_n \in C$ $(n \gg_g 1)$, and a divisor $D_0 \gg_g 0$. Let $V = \L(2D_0)$.

\vspace{5mm}

\item $[D-D_0] \in J$ is represented by $\L(2D_0-D) \subset V$.

\vspace{5mm}

\item An element $v \in V$ is represented by $v(P_1), \cdots, v(P_n)$. 

A subspace of $V$ is represented by a basis.

\end{itemize}

\vspace{2mm}

\end{block}

\end{frame}

\begin{frame}
%\frametitle{Any questions ?}

\begin{center}
\huge Thank you !
\end{center}

\end{frame}

\end{document}
